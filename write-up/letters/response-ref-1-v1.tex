%%%%%%%%%%%%%%%%%%%%%%%%%%%%%%%%%%%%%%%%%
% Plain Cover Letter
% LaTeX Template
%
% This template has been downloaded from:
% http://www.latextemplates.com
%
% Original author:
% Rensselaer Polytechnic Institute (http://www.rpi.edu/dept/arc/training/latex/resumes/)
%
%%%%%%%%%%%%%%%%%%%%%%%%%%%%%%%%%%%%%%%%%

%----------------------------------------------------------------------------------------
%	PACKAGES AND OTHER DOCUMENT CONFIGURATIONS
%----------------------------------------------------------------------------------------

\documentclass[11pt]{letter} % Default font size of the document, change to 10pt to fit more text

\usepackage{newcent} % Default font is the New Century Schoolbook PostScript font
%\usepackage{helvet} % Uncomment this (while commenting the above line) to use the Helvetica font

% Margins
\topmargin=-1in % Moves the top of the document 1 inch above the default
\textheight=8.5in % Total height of the text on the page before text goes on to the next page, this can be increased in a longer letter
\oddsidemargin=-10pt % Position of the left margin, can be negative or positive if you want more or less room
\textwidth=6.5in % Total width of the text, increase this if the left margin was decreased and vice-versa

\let\raggedleft\raggedright % Pushes the date (at the top) to the left, comment this line to have the date on the right
\usepackage{amsthm, amsfonts}
\usepackage{hyperref}


\def\Y{{\bf Y}}

\begin{document}

%----------------------------------------------------------------------------------------
%	ADDRESSEE SECTION
%----------------------------------------------------------------------------------------

\begin{letter}{Professor
	Peter Glynn\\
	Editor, {\em Journal of Applied Probability}}

%----------------------------------------------------------------------------------------
%	YOUR NAME & ADDRESS SECTION
%----------------------------------------------------------------------------------------

%\begin{center}
%\large
%\end{center}
%\vfill

\signature{Walter Dempsey\\
University of Michigan\\
Department of Biostatistics\\
1415 Washington Heights\\
Ann Arbor, MI 48103} % Your name for the signature at the bottom

%----------------------------------------------------------------------------------------
%	LETTER CONTENT SECTION
%----------------------------------------------------------------------------------------

\vspace{5mm}

\newpage

{\bf Response to Referees}

Let us start by saying that these are the most thorough and helpful reviews we have ever received on a submitted manuscript.  We appreciate all of the very helpful comments from the editor, associate editor, and the two referees which has helped improve the paper substantially. We have addressed the concerns as best as possible in this revision. We provide a further point-by-point explanation below. Any comments in {\it italics} indicate a statement from the referee report.

{\bf Response to Editor and Associate Editor}
\begin{enumerate}
\item {\it The referees have many specific concerns they are asking to address, and their expertise and viewpoints are important to revise this manuscript to improve its clarity and impact.}

\vspace{5mm}
We thank the editor-in-chief as this is an excellent point.  We now acknowledge that both topics are well known and cite relevant literature in Section 1.1; however, while infectious disease researchers and survey statisticians have been aware of the perils of self-selection and measurement error, the interplay in this context has been less well studied.   By extending the statistical decomposition of Meng (2018), we help shed light on how measurement error does not act to simply alter the magnitude of error but may in fact change the sign as well.  This decomposition motivates the new inverse-probability and doubly robust methodologies in the revised manuscript.
\vspace{5mm}

\item {\it Some of these deal with the fast-evolving nature of the pandemic with new information emerging every week and changing variants with different characteristics also changing our understanding of the disease course and spread and timing, that affect the content in the paper. I understand that it is not feasible for a peer-reviewed paper like this to include all the latest information in a fast-changing environment like this in which understanding can change quickly, but it is important in this revision to update the KEY points as much as possible based on updated knowledge, but otherwise revise the language to make it “forward compatible” in terms of relevance as things continue to change, e.g. by acknowledging some details of spread dynamics may change with future variants, but presenting the current work based on a particular SARS-CoV-2 variant as illustrative of the issues that arise with each new variant and other viruses, as well. That is, recognizing it may not be feasible to keep updating based on emerging knowledge from each new strain, but to lock down the discussion context but add enough explanation so the reader can see its relevance for new strains.}

\vspace{5mm}
x
\vspace{5mm}

\item {\it The AE also mentions a few comments as well:}
\begin{itemize}
	\item {\it Title: As noted by one of the reviewers, the “paradox” is never discussed/defined in the manuscript. Consider changing to “Statistical estimation of Coronavirus Case Counts: Measurement…” or some alternative.}
	\item {\it Organization: The figures and tables are not located anywhere near much of their introduction/discussion. For example, Table 1 was discussed on page 7 and is itself located on page 22. Consider the reorganization such that Figures and Tables are located near their discussion.}
	\item {\it Discussion on the Indiana Random Sample: On page 22, table 1, it appears that the random sample is not similar to the census estimates, potentially indicating non-response bias.}
	\item {\it Additionally, there may be significant potentially non-ignorable variables (e.g., political opinions and rural status) that drive non-response and are not presented. Is there any comment/discussion on the success of the randomization to be used as the “gold-standard” in this example? Much of the results appear to rely on the SRS and it ignores the potential for nonresponse bias.}
\end{itemize}

\vspace{5mm}
Thank you for raising this issue.  We discuss reporting delays and the relevant literature in Remark 1 in Section 2.2.  Given currently available public data, one cannot correct at the daily level for reporting delays.  Instead, we analyze COVID-19 data at the weekly level to account for potential reporting delays.  While not a perfect solution, it avoids clear reporting variation across day-of-week which can be seen in Figure 5.
\vspace{5mm}
\end{enumerate}
\newpage
{\bf Response to Reviewer 1}
I am incredibly grateful for this review. (PRAISE OUT THE WAZZOOO).

\begin{enumerate}
\item {\bf Section 1: Introduction.} {\it It may be useful in the introduction to briefly discuss the various purposes of testing for diseases – for COVID we have largely focused on diagnostic testing of people who are symptomatic or have a known or suspect exposure. Diagnostic testing will never provide valid prevalence or incidence estimates, especially for a disease where a significant proportion of cases are asymptomatic or pauci-symptomatic and thus unlikely to present for diagnostic testing. Screening tests are more appropriate as a potential tool for reconstructing prevalence or incidence tests, although these are seldom conducted at random. The random and non-random testing samples appear to represent screening and diagnostic testing goals, respectively, and this is part of why they can be expected not to deliver the same prevalence estimates.}

\vspace{5mm}
Section XX now discusses the purposes of testing.
\vspace{5mm}

\item {\bf Section 1.1}
\begin{itemize}
	\item {\it For the first goal of the paper, it might be helpful to link the biases with those outlined in a set of papers from the Harvard Center for Disease Dynamics: \url{https://doi.org/10.1007/s10654-021-00727-7} and DOI:\url{10.1093/aje/kwaa188}}
	\vspace{5mm}
	We now connect with XX
	\vspace{5mm}
	\item {\it For the second goal of the paper, it would be useful to discuss how the proposed method differs from the more traditional epidemiologic method of obtaining a validation sample. As I see it, the chief distinction is that here the “gold standard” is not a different test (although one could argue that the gold standard being used here is a screening test), but rather it is a less biased selection mechanism. Clarifying the links with validation studies could be helpful for epidemiologists \& statisticians alike, as the use of a validation study to correct for selection bias is a fairly novel suggestion.}
	\vspace{5mm}
	Remark within validation subsection to contrast with validation sample.
	\vspace{5mm}
	\item {\it Finally, consider linking the arguments you are making with recent work on ‘target validity’ (see: \url{https://doi.org/10.1093/aje/kwy228}). Literature on target validity has largely been focused on specifying the problem of tackling internal and external validity together, but the method you propose here is a concrete attempt to address both types of validity and extends this conversation nicely.}
	\vspace{5mm}
	Validation subsection
	\vspace{5mm}
\end{itemize}
\item {\bf Section 2.1: Diagnostic testing}
\begin{itemize}
	\item {\it Please clarify in the first paragraph the distinction between incubation period (time to symptom onset) and latent period (time to infectiousness). The viral load will be detectable by at least the end of the latent period which for SARS-CoV-2 occurs before the end of the incubation period. However, individuals are unlikely to present for diagnostic testing until after the end of the incubation period.}
	\vspace{5mm}

	We now connect with XX
	\vspace{5mm}
	\item {\it Please also clarify the terminology ‘active infection’ which in some instances appears to refer to individuals in the symptomatic period of infection, and in other instances appears to refer to individuals in the infectious period of infection. We know that these are not identical in SARS-CoV-2 infection and that difference is a key reason why screening and diagnostic tests are not easily comparable.}
	\vspace{5mm}

	We clarify that here we are interested in all individuals who .


	\vspace{5mm}
	\item {\it Consider adding some additional information on screening tests, particular anything relevant to the random sample conducted in Indiana which is used as the validation set. Were the screening tests conducted using nasopharyngeal swabs (in many places screening tests use nasal swabs instead and often self-collected samples)? Were the molecular tests used to assay the samples the same for screening and diagnostic tests? Differences in sample collection and testing processes could undermine the utility of this random sample as a validation set.}
	\vspace{5mm}

	We clarify the means. These were individuals who report to the exact same set of locations as in the non-random sample.  So they are representative.
	Note, we do not use the outcome data in the method for validation, only for adjustment for self-selection in the diagnostic tests.
	\vspace{5mm}
\end{itemize}
\item {\bf Section 2.1}
\begin{itemize}
	\item {\it This section could benefit from a few remarks on why it is important / useful to focus on publicly available testing data. There are certainly distinct statistical issues, if public data are aggregated, but are the selection bias and measurement error issues different between public and government datasets? If so, how do the solutions presented here apply to government datasets? Can similar or the same approaches be used by health departments to assess the pandemic and guide responses?}
	\vspace{5mm}

	Clarify that Indiana released a lot of aggregate information.  Two distinct goals: (1) what can we say with additional surveys but not additional testing, (2) guide health departments who have access to
	\vspace{5mm}
	\item {\it A timeline of restrictions and testing eligibility would be a useful addition to Section 2.2.1. Also, what there no time period where testing in Indiana was restricted on the basis of travel history or suspected exposure?}
	\vspace{5mm}

	We have a timeline
	\vspace{5mm}
\end{itemize}
\item {\bf Section 2.3: Probabilistic Samples}
\begin{itemize}
	\item {\it The $\approx 30\%$ response rate in the supposedly random sample requires some remarks. Even if the original list of eligible individuals was random, it seems unlikely that the final sample remains random with such low response. Is it possible to describe any characteristics that may differ between the intended sample and the final sample? Certainly the final random sample differs in many ways from the Census data presented in Table 1. Also, please clarify if the Census data is US total or Indiana specific. }
	\vspace{5mm}

	We can discuss the non-response weights used in their paper to adjust.  And that this is a different and solvable issue (we can just weight by the )
	\vspace{5mm}
	\item {\it For the Facebook / CMU survey sample, please present the age breakdown of sampled individuals. It is well-known that the age-distribution of Facebook users differs from the general population, so this is unlikely to be a representative sample of the Indiana population. Other demographic information on this sample would be valuable too. (Why is it not included in Table 1?)}
	\vspace{5mm}

	We include an age breakdown of the data.  Note that weights are used to adjust and we are using these in the calculations.  We now include a table in the appendix a breakdown of the demographic information for each sample.
	\vspace{5mm}
\end{itemize}
\item {\bf Section 3: Analysis of case count data}
\begin{itemize}
	\item {\it Please clarify the definition of ‘active infection’ (any infection, vs infectious, vs symptomatic, vs detectable by molecular testing?). This has important implications for the construction of the compartmental model and for the definition of your period prevalence (AIR).}
	\vspace{5mm}

	XXXXX
	\vspace{5mm}
	\item {\it Throughout, there is a general issue of too few distinct variable identifiers – for example, using Y to refer to infection status and y to refer to test status will likely lead to confusion.}
	\vspace{5mm}

	We clarify our choice of notation by adding an appendix with notation.
	\vspace{5mm}

	\item {\it The “natural candidate for AIR” presented is the quantity that has more commonly been referred to as the “test positivity rate” (assuming that yi does in fact indicate test status rather than infection status which is obviously unknown). It would be helpful to mention this.}
	\vspace{5mm}

	This is now stated explicitly
\end{itemize}
\item {\bf Section 3.1: Imperfect testing}
\begin{itemize}
	\item {\it Here, the value of distinguishing between test status and infection status becomes clearer. By using $P_j$ to mean ‘error status of testing’, the derivations in this section lead to a focus on false positives and false negatives. However, the real quantity of interest both for individuals and for determining prevalence is surely the positive predictive value and the negative predictive value – that is, Pr($Y_j$=1|positive test result) and Pr($Y_j$=0|negative test result). Unlike the FP and FN which (because they are strictly functions of sensitivity and specificity) are only affected by test characteristics, the PPV and NPV are affected also by prevalence. Acknowledging this leads more directly to the benefit of an iterative method for validating the test results (since FP \& FN should not depend on Y or y \& thus not need an iterative method).}
	\vspace{5mm}

	The adjustment is exactly related to the Bayes rule PPV/NPV relationship with FP, FN, and $\bar Y$.
	\vspace{5mm}
	\item {\it Also worth noting that test positivity is not strictly binary (since it’s generally based on an underlying continuous cycle threshold), and so measurement error could potentially lead towards bias away from, or across, the null even in the absence of selection bias.}
	\vspace{5mm}

	We now include a discussion of the underlying continuous cycle (with proper credit).  The binary variable of interest can depend on infectious is what is often most important.  So we highlight that most but acknowledge the binary variable can be defined....
	\vspace{5mm}

	\item {\it The estimator $y = (yn-FP) /(1-(FP+FN))$ is a standard estimator for correcting for measurement error using FP \& FN estimated from a validation sample, where yn is the mis-measured average \& y is the corrected average. The use of this estimator to iteratively estimate the point prevalence is a good contribution of this paper, but the failure to distinguish between infection \& test status in the variable names somewhat obscures that this is what the author is doing.}
	\vspace{5mm}

	We now more carefully distinguish.  And yes, this is the standard estimator.  We have simply derived it from a new perspective (the decomposition perspective which didn't require any Bayesian machinery).
	\vspace{5mm}

	\item {\it Finally, in the section on effective sample size, please clarify why the FP and FN would be dependent on testing capacity. False positive and false negative rates, as well as sensitivity and specificity, are strictly characteristics of a given test – if increased test capacity or burden would alter the sample collection quality then FP \& FN could change. But if increased test capacity simply changes the underlying prevalence in the group tested (because of opening up testing to uninfected individuals), then the FP \& FN should not change, but the PPV and the NPV will change since these are dependent on prevalence.}
	\vspace{5mm}

	I now give an explicit example.  The test characteristics may change based on the nurse (cite) and/or what exact test is used.  For example, a center that moves from nasal swab to spit to increase testing capacity will see a ...(cite).
\end{itemize}
\item {\bf Section 3.2: Regrettable Rates}
\begin{itemize}
	\item {\it Here, and throughout the theoretical example, the model is an SIR model. However, we know that SARS-CoV-2 is better represented with an SEIR model, and indeed the applied example later in the manuscript uses an SEIR approach. Since this paper is very particularly addressing an issue with COVID data it would be beneficial to use an SEIR model throughout rather than the simpler SIR model.}
	\vspace{5mm}

	The SIR model in Section 3.2 is replaced by the SEIR model.  We highlight that the SEIR model has been shown to better represent the SARS-CoV-2 dynamics.  Figure 3a presents updated versions of the potnetial bias in ratio estimators under an SEIR model.   The conclusion remains largely the same. The only difference from the prior manuscript is that the peak in bias prior to the ratio going less than 1 is less pronounced (was up to 5, now up to 2).
	\vspace{5mm}
\end{itemize}
\item {\bf Section 3.3: Effective Reproduction Number}
\begin{itemize}
	\item {\it Here again, an SEIR model would be valuable. It is also important to clarify that the effective reproductive number can be defined in a number of subtly different ways, and that these impact both the method of estimation and the interpretation of the effect. See for example Cori et al; Am J Epi, 178(9):1505-12. 2013 and Wallinga \& Tuenis Am J Epi; 160(6):509-16. 2004 for methods that, while similar, estimate subtly different quantities. It would be helpful to clarify exactly what effective reproductive number is being estimated here – from the formulae, I suspect it is the type described by Cori et al rather than Wallinga \& Tuenis.}
	\vspace{5mm}

	We clarify the subtle differences between the definitions of the instantaneous and case reproductive numbers.  We establish that our approach aims to estimate the instantaneous reproduction number, i.e., is closely related to the proposal by Cori et al. (2013)).

	\vspace{5mm}
	\item {\it The Rt estimator will differ if an SEIR model is used, and it is not clear (to me) that the same intuition about the error will hold.}
	\vspace{5mm}

	%https://academic.oup.com/aje/article/189/11/1438/5864525?searchresult=1, plot SIR
	%https://academic.oup.com/aje/article/178/9/1505/89262?searchresult=1, CORI
	%https://www.nature.com/articles/s41598-020-76563-8.pdf?origin=ppub, Althaus approximate

	This estimator is originally motivated by the SIR model and is exact under the poisson likelihood; however, the moment-based estimator has been proposed as a general tool for assessing epidemic trends in real time (Xinhua Yu, Assessing Epidemic Trends in Real Time With a Simple Ratio Plot, Am J of Epid, 189(11), 2020).  The estimator is closely related to the instantaneous reproduction numbers proposed by Cori et. al. (2013), but we now rely on recent work (Heng and Althaus, The approximately universal shapes of epidemic curves in the SEIR model, Sci Reports, 2020) to propose an approximate moment-based estimator in the SEIR setting. Specifically, Heng and Althaus (2020) derive the following approximate formula for the basic reproduction number under SEIR dynamics:
	$$
	R_0 = \frac{1 + \Lambda \left(\frac{1}{\sigma} + \frac{1}{\gamma} \right)}{S_0 + I_0 (1 + \gamma/\sigma)}.
	$$
	where~$\Lambda$ is the epidemic growth rate, and $S_0$/$I_0$ are the initial fractions of the population that are susceptible and infectious respectively.  For $S_0 \approx 1$ and $I_0 \ll 1$, we have the approximate relationship $R_0 \approx 1 + (\frac{1}{\gamma} + \frac{1}{\sigma})\Lambda$.
	, this is an approximate solution which lacks the term $\Lambda^2 \frac{1}{\gamma \sigma}$.  ``It corresponds to the case of an exponentially distributed generation time with mean duration $1/\gamma + 1/\sigma$, which is the same as the solution for the SIR model assuming an infectious period of $1/\gamma + 1/\sigma$. For realistic growth rates and choices of $\sigma$ and $\gamma$ for the COVID-19 pandemic, this term is negligible and therefore we work with the approximate formula.
	\vspace{5mm}

	\noindent Note that for the SIR model, the basic reproduction number is~$R_0 = 1 + \frac{1}{\gamma} \Lambda$. So estimate of the instantaneous reproduction number is obatined by replacing $\Lambda$ by an estimate of the  instantaneous growth rate $\hat \Lambda_t = \log (K_t / K_{t-1})$.  Then using the approximate relationship between the SEIR model and an SIR model with infectious period $1/\gamma + 1/\sigma$, we arrive at the new moment-based estimator
	$$
    R_t \approx 1 + \left( \frac{1}{\gamma} + \frac{1}{\sigma} \right) \log \left( \frac{K_t}{K_{t-1}} \right).
	$$
	The updated $R_t$ estimator shows that the same intuition about the error holds when we are considering an SEIR rather than an SIR model.  Figure 3b shows the reproduction number estimates and potential biases under SEIR dynamics which shows the same general conclusions hold.  For completeness, we also present the instantaneous reproduction number estimates using the proposal of Cori et. al. (2014) in Section I of the supplementary materials, which shows the same general conclusions.
\end{itemize}
\item {\bf Section 3.4: Rate comparisons}
\begin{itemize}
	\item {\it This section would also be more relevant with an SEIR model}
	\vspace{5mm}

	This section has been also updated to consider an SEIR model.  All conclusions remain the same.
	\vspace{5mm}
\end{itemize}
\item {\bf Section 4.1: Selection propensity estimation}
\begin{itemize}
	\item {\it The use of weights to correct for sampling bias is an established practice in epidemiology. It would be helpful to discuss how the proposed method differs from survey sampling weights or transportability weights (see Am J Epidemiol 2017 Oct 15;186(8):1010-1014. doi: 10.1093/aje/kwx164.).}
	\vspace{5mm}

	Inverse probability of selection weights describe by Cole and Stuart are exactly what we are building.  However, how to build them is the question.
	As far as we are aware, the \href{approaches}{https://academic.oup.com/aje/article/172/1/107/82724} require knowledge of the selection mechanism which is not available in our setting.  Instead, we require auxiliary information.  Here, (cite elliott stuff).  A key different is the time-varying nature which we highlight.
	\vspace{5mm}
	\item {\it This section, and the sections below, also become somewhat confusing to follow because of the multiple different variables which are all denoted Ij. Since a chief goal is to describe and correct for errors in the sampling process, it would be much more useful for the reader if there were different notation for each of the different sampling mechanisms being discussed / used (eg, inclusion in the random sample, inclusion in the non-random validation sample, inclusion in the original diagnostic testing data, inclusion in ?? on day x).}
	\vspace{5mm}

	We have no built a notation table and made the notation different.  Apologies for the confusion. We use $I_j^{R}$ and $I_j^{NR}$.
	\vspace{5mm}
	\item {\it In the final formula (eq 4.3), I am not clear on why the weights are applied to both yk and FP (ie why sum(w*(yk-FP)) and not sum(w*yk) – FP)? This is very likely just me not understanding the derivation in Appendix C, but some explanation of the intuition would be helpful.}
	\vspace{5mm}

	Since we normalizing the weights, this is just to write it in simple form but basically $\sum w y$ is the estimate and we subtract FP.  So you are right!
	\vspace{5mm}
	\item {\it The time-varying section seems somewhat unnecessary since it is simply taking the expectation of the data over some time period – the daily data also does this, but where the time-period is 1 day. }
	\vspace{5mm}

	Necessary for smoothing.
	\vspace{5mm}
\end{itemize}
\item {\bf Section 4.2: model-based estimation}
\begin{itemize}
	\item {\it The SEIR model described in the text and in the supplement differ in a way that I find confusing – the code and equations describe a simple SEIR model but the written text describes an SEmInR model with stratum-specific covariates which appear to apply to at least the I compartments and possibly the E compartments. From the code and equations, presumably the assumption is that all strata mix with identical probability and all strata experience the same probabilities of infection, exposure, and transmission. However, these assumptions are (1) unlikely to hold and (2) make it hard to understand why the model includes strata at all. }
	\vspace{5mm}

	Prior is Dirichlet (uniform) but not time-varying.  This allows us to learn the links.
	\vspace{5mm}
\end{itemize}
\item {\bf Section 4.3: Doubly Robust Estimation}
\begin{itemize}
	\item {\it The term $\mu(x_j)$ in the unnumbered equation does not seem to be defined anywhere – this is presumably coming from the SEIR model given that the rest of the equation is essentially the IPW-based estimator. However, it is unclear to me what output from the SEIR this term includes.}
	\vspace{5mm}

	The SEIR model is used to generate the number of new weekly cases which allows us to calculate the active infection rate.  $\mu(x_j)$ is the model-based estimate.
	\vspace{5mm}
\end{itemize}
\item {\bf Section 5: Applied analysis}
\begin{itemize}
	\item {\it The applied example relies heavily on the idea that the random sample is a true random sample of the population of Indiana eligible to be infected with COVID, and that the non-random sample and the Facebook / CMU sample similarly provide representative measures of covariate data. This is not well-established by the short data descriptions given and deserves further emphasis.}
	\vspace{5mm}

	We provide further emphasis and evidence.  A key point is that this data is used to illustrate the 'what to do' point that was raised in the first round.
	\vspace{5mm}
\end{itemize}
\item {\bf Section 5.1.1. Disease Prevalence}
\begin{itemize}
	\item {\it This section seems to assume that individuals can only be tested once but that assumption is unlikely to be true. In particular, people with higher exposures will get tested more frequently than people with lower exposures, and similarly people at higher risk of severe outcomes may be tested more frequently than people with lower risk. Calculating the sampling fraction as the (number of tests)/(total pop – previously tested) thus seems likely to provide a biased sampling fraction estimate.}
	\vspace{5mm}

	We acknowledge this limitation.  As an alternative we assume the risk set is always total pop.  This presents a range.
	\vspace{5mm}
	\item {\it Also, it would be useful to clarify whether the random sample, which is being used as the gold standard, was corrected for measurement error, as well as how the issue of 30\% response rate was dealt with in obtaining the prevalence estimate. Given these issues, the random sample estimate could easily also be biased.}
	\vspace{5mm}

	Cite PNAS for measurement error and response rate cleaning.
	\vspace{5mm}
\end{itemize}
\item {\bf Section 5.1.2 Time varying estimate}
\begin{itemize}
	\item {\it 5.1.2. In the second approach, using hospitalization data to attempt to back-predict fever status seems highly likely to be subject to survivor bias as well as biases driven by differential access to care. Some discussion of the issues with this approach is warranted.}
	\vspace{5mm}

	We have added Remark XX to discuss the bias.  In Remark XX, we state that availability of symptom status (which we know the state collects but has told us is unavailable due to data privacy / confidentiality issues) is important to use.
	\vspace{5mm}
	\item {\it Also, focusing on the non-Hispanic, white male groups seems potentially to limit understanding of the scope of error. Although Indiana has a very high percent of white individuals, the individuals who present for testing (as well as those who test positive, are exposed in the workplace and home, and hospitalized or die) are disproportionately non-white. It is important to show the performance of the method in both white and non-white individuals because non-white individuals are the ones who are experiencing higher rates of illness and death and therefore will likely be more negatively impacted by biased estimates of disease prevalence.}
	\vspace{5mm}

	We acknowledge this shortcoming.  We now include versions of Figure 6 for non-white fr
	\vspace{5mm}
\end{itemize}
\item {\bf 5.2 SEIR model}
\begin{itemize}
	\item {\it As stated above, some description of the assumptions related to contact patterns and transmission between and among age strata is needed. In addition, please clarify which compartments of the model the infection-fatality rate applies to (E \& I or I only), as well as how the model handles death from non-COVID causes from these and other model compartments.}
	\vspace{5mm}

	We now describe the model in more detail.
	\vspace{5mm}
	\item {\it Also, the discussion of IFR would benefit greatly from being contextualized to the rest of the paper – the goal of the manuscript is specifically to provide a better estimate of who is infected. Such an estimate would therefore be expected to provide a more valid infection fatality rate, and this in turn implies the IFR used here must be biased. This should be discussed.}
	\vspace{5mm}

    We have now acknowledged this along with the fact that IFR is time-varying. So this shows the IFR is likely biased (or was only accurate early).  We performed sensitivity analysis with time-varying IFR based on estimates to improve estimation.
	\vspace{5mm}
\end{itemize}
\end{enumerate}
\newpage

{\bf Response to Reviewer \#2}

I am incredibly grateful for this review.  I agree that the paper has evolved beyond the current title.  This was not updated in the prior submission simply to ensure continuity of presentation (i.e., didn't want to change from first submission).  Here we replace with the title ``Good Title''.  We next address the reviewers specific comments.

\begin{itemize}
\item {\it The paper states some facts about the SARS-CoV-2 virus that are outdated, for example “Upon infection with SARS-CoV-2, an incubation period starts and lasts approximately five days [Lauer et al., 2020] before the viral load is high enough to be detected.” (Page 5). Lauer was based on the original
strain, we see different disease dynamics with the delta strain. Perhaps to be more precise, the author could clarify that this is an example of the original SARS-CoV-2 incubation rather than definitative as there will likely be future variants that will change this further. Similarly, on page 13, the serial
interval of 7 is stated as fact – this is again based on the original strain rather than the current disease dynamics. Stating these numbers as an example rather than definitive would be preferable, I believe.}
\vspace{5mm}

We have added remark XX and more carefully put this paper within proper context.
\vspace{5mm}
\item {\it The author states: “If someone has an active infection then after the incubation period, a molecular test with perfect sensitivity will yield a positive result for the next few weeks.” (Page 5) The author states “A molecular test refers to diagnostic tests that aim to detect increased viral load” (page 5) - I think a more precise deinition is needed, for example aim to detect viral load above a certain threshold.}
\vspace{5mm}

We use Figure XX to make precise the importance of viral threshold.
\vspace{5mm}
\item {\it The author states “If someone has an active infection then after the incubation period, a molecular test with perfect sensitivity will yield a positive result for the next few weeks.” (page 5) I’m not sure it is accurate that the patient will have a positive result for the next few weeks, perhaps “If someone has an active infection then after the incubation period, a molecular test with perfect sensitivity will yield a positive result while the patient has a viral load above the threshold of detection”.}
\vspace{5mm}

We thank the reviewer and now match this more careful language to be more exact in our description.
\vspace{5mm}
\item {\it The author lumps RT-PCR and antigen tests in together and discusses that a patients will likely have positive tests for several weeks while not infectious (able to transmit) – for the most part this is the case for RT-PCR tests, however antigen tests give a clearer picture of infectiousness as the viral threshold is higher and often correlates with the likelihood to transmit.}

\vspace{5mm}
We now make this distinction much clearer.
\vspace{5mm}
\item {\it XX}

\vspace{5mm}
WXX
\vspace{5mm}
\item {\it I’m not sure this statement “The primary goal of molecular tests is diagnosis of active infections in the population.” (page 5) is entirely true. I definitely think this is a worthwhile goal, but many in the medical community use molecular tests to determine the diagnosis of an individual patient, regardless if the infection is “active” or not (i.e. whether they are currently infectious rather than infected which is what I assume the author means by “active”).}

\vspace{5mm}
WXX
\vspace{5mm}
\item {\it On page 13, the author discusses the bias relative to the “peak” – is this the peak in cases? I believe so, but it would be nice to define a bit more clearly.}

\vspace{5mm}
WXX
\vspace{5mm}
\item {\it We would expect the IFR to dramatically change after the introductions of vaccines in December 2020, making this method less applicable without introducing a time-varying component.}

\vspace{5mm}
WXX
\vspace{5mm}
\item {\it I found Figure 3 in the supplement conusing. Figure 3A shows the profile for a 35-44 white male with symptoms and COVID-19 contact and Figure 3B shows the profile for a 65-74 white female without symptoms nor COVID-19 contact – I would have expected the covariates between these (age, ethnicity, race, gender) to be the same rather than differ with the only thing differing being symptoms 1/ COVID-19 contact? As a minor note, the y-axis labels do not match for this plot.}

\vspace{5mm}
WXX
\vspace{5mm}
\item {\it The model based estimates in Figure 8 seem so different from the other estimates that it is potentially implausible and suggests misspecification of the IFR. Is this potentially due to the age distribution of cases and the IFR collapsing choices? Perhaps as a sensitivity analysis different IFR values should be examined}

\vspace{5mm}
WXX
\vspace{5mm}
\item {\bf Minor comments:}
	\begin{enumerate}
		\item {\it Figure 3 mention “$\beta = 1.2$ (black)” – I’m not sure what “black” refers to?}
		\vspace{5mm}

		We have removed this and simply state the SEIR parameter values ($\beta = 1.2, \gamma = 0.14, \sigma = 0.3$) underlying the Figure.

		\item {\it In the supplement, page 21, the author refers to “Figure 6a and 6a” – I believe they mean “Figure 6a and 6b”. This also claims to show the likelihood for all age ranges - I believe it is only for males.}
	\end{enumerate}
\end{itemize}

\end{letter}






\end{document}





