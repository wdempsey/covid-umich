%%%%%%%%%%%%%%%%%%%%%%%%%%%%%%%%%%%%%%%%%
% Plain Cover Letter
% LaTeX Template
%
% This template has been downloaded from:
% http://www.latextemplates.com
%
% Original author:
% Rensselaer Polytechnic Institute (http://www.rpi.edu/dept/arc/training/latex/resumes/)
%
%%%%%%%%%%%%%%%%%%%%%%%%%%%%%%%%%%%%%%%%%

%----------------------------------------------------------------------------------------
%	PACKAGES AND OTHER DOCUMENT CONFIGURATIONS
%----------------------------------------------------------------------------------------

\documentclass[11pt]{letter} % Default font size of the document, change to 10pt to fit more text

\usepackage{newcent} % Default font is the New Century Schoolbook PostScript font
%\usepackage{helvet} % Uncomment this (while commenting the above line) to use the Helvetica font

% Margins
\topmargin=-1in % Moves the top of the document 1 inch above the default
\textheight=8.5in % Total height of the text on the page before text goes on to the next page, this can be increased in a longer letter
\oddsidemargin=-10pt % Position of the left margin, can be negative or positive if you want more or less room
\textwidth=6.5in % Total width of the text, increase this if the left margin was decreased and vice-versa

\let\raggedleft\raggedright % Pushes the date (at the top) to the left, comment this line to have the date on the right
\usepackage{amsthm, amsfonts}
\usepackage{xcolor}
\usepackage{hyperref}


\def\Y{{\bf Y}}

\begin{document}

%----------------------------------------------------------------------------------------
%	ADDRESSEE SECTION
%----------------------------------------------------------------------------------------

\begin{letter}{Professor
	Peter Glynn\\
	Editor, {\em Journal of Applied Probability}}

%----------------------------------------------------------------------------------------
%	YOUR NAME & ADDRESS SECTION
%----------------------------------------------------------------------------------------

%\begin{center}
%\large
%\end{center}
%\vfill

\signature{Walter Dempsey\\
University of Michigan\\
Department of Biostatistics\\
1415 Washington Heights\\
Ann Arbor, MI 48103} % Your name for the signature at the bottom

%----------------------------------------------------------------------------------------
%	LETTER CONTENT SECTION
%----------------------------------------------------------------------------------------

\vspace{5mm}

\newpage

{\bf Response to Referees}

Let us start by saying that these are the most thorough and helpful reviews we have ever received on a submitted manuscript.  We appreciate all of the very helpful comments from the editor, associate editor, and the two referees which has helped improve the paper substantially. We have addressed the concerns as best as possible in this revision. We provide a further point-by-point explanation below. Any comments in {\it italics} indicate a statement from the referee report.

{\bf Response to Editor and Associate Editor}
\begin{enumerate}
\item {\it The referees have many specific concerns they are asking to address, and their expertise and viewpoints are important to revise this manuscript to improve its clarity and impact.}

\vspace{5mm}
We thank the editor-in-chief.  We have addressed the two reviewers' specific comments.  Responses to their concerns are detailed below and have significantly improved the clarify of this manuscript.
\vspace{5mm}

\item {\it Some of these deal with the fast-evolving nature of the pandemic with new information emerging every week and changing variants with different characteristics also changing our understanding of the disease course and spread and timing, that affect the content in the paper. I understand that it is not feasible for a peer-reviewed paper like this to include all the latest information in a fast-changing environment like this in which understanding can change quickly, but it is important in this revision to update the KEY points as much as possible based on updated knowledge, but otherwise revise the language to make it “forward compatible” in terms of relevance as things continue to change, e.g. by acknowledging some details of spread dynamics may change with future variants, but presenting the current work based on a particular SARS-CoV-2 variant as illustrative of the issues that arise with each new variant and other viruses, as well. That is, recognizing it may not be feasible to keep updating based on emerging knowledge from each new strain, but to lock down the discussion context but add enough explanation so the reader can see its relevance for new strains.}

\vspace{5mm}
Remark 1 has been added at the end of the introduction to contextualize the manuscript. Specifically we clarify that our manuscript focuses on COVID-19 case count, testing, and death data collected from April 2020 through February 2021 and therefore the numbers presented on disease dynamics are based on the original strain. While using the framework to account for selection bias and measurement error for data arising from the delta and omicron strains would require additional work, the framework would remain an appropriate tool for this task.
\vspace{5mm}

\item {\it The AE also mentions a few comments as well:}
\begin{enumerate}
	\item {\it Title: As noted by one of the reviewers, the “paradox” is never discussed/defined in the manuscript. Consider changing to “Statistical estimation of Coronavirus Case Counts: Measurement…” or some alternative.}
	\vspace{5mm}

	We agree with the AE. This was not updated in the prior submission simply to ensure continuity of presentation (i.e., didn’t want to change from first submission). Here we replace with the title ``Addressing selection bias and measurement error in COVID-19 case count data using auxiliary information.''
	\vspace{5mm}

	\item {\it Organization: The figures and tables are not located anywhere near much of their introduction/discussion. For example, Table 1 was discussed on page 7 and is itself located on page 22. Consider the reorganization such that Figures and Tables are located near their discussion.}
	\vspace{5mm}

	We apologize.  We kept figures/tables related to the case study in Section 5 (the case study section) but can now see this led to confusion as we discuss them early in the paper. Tables and figures have been moved to be located much closer to their introduction.
	\vspace{5mm}

	\item {\it Discussion on the Indiana Random Sample: On page 22, table 1, it appears that the random sample is not similar to the census estimates, potentially indicating non-response bias.}
	\vspace{5mm}

    See our answer to Question 5a from Reviewer \#1. We now remark on the response rate in Section 5.  Specifically, we cite how Yiannoutsos et al. (2021) handle non-response in prevalence estimation.  We acknowledge the potential for a difference in characteristics between intended and final sample.
    \vspace{5mm}


	\item {\it Additionally, there may be significant potentially non-ignorable variables (e.g., political opinions and rural status) that drive non-response and are not presented. Is there any comment/discussion on the success of the randomization to be used as the “gold-standard” in this example? Much of the results appear to rely on the SRS and it ignores the potential for nonresponse bias.}
	\vspace{5mm}

	We do not see random sampling as a ``gold-standard'' nor a panacea that will fix every issue.  Instead, we see it as a means to remove one very important source of potential bias in the estimation of population quantities of interest.  A key takeaway of our paper is that with access to \emph{only} a non-probabilistic sample, we know there is potential bias but have no way to reduce said bias.  With a probability sample, we have a means to do so.  Statisticians, epidemioligists, government officials, and health policy experts can then focus on practical issues such as addressing non-ignorable missingness.  If we thought, for example, political opinions and rural status are important predictors of missingness, then these should be collected as part of the random sampling protocol.  Note, even collecting strata information that correlates with testing propensity (e.g., symptom status) in the non-probabilistic sample is insufficient to address selection bias.  This is not true for probability samples.  We have added this important discussion to Section 6.
\end{enumerate}

\end{enumerate}
\newpage
{\bf Response to Reviewer 1}
\vspace{5mm}

I am very grateful for this review.  The detailed suggestions and comments helped improve the paper in numerous ways.  See below for our responses to each particular reviewer comment.

\begin{enumerate}
\item {\bf Section 1: Introduction.} {\it It may be useful in the introduction to briefly discuss the various purposes of testing for diseases – for COVID we have largely focused on diagnostic testing of people who are symptomatic or have a known or suspect exposure. Diagnostic testing will never provide valid prevalence or incidence estimates, especially for a disease where a significant proportion of cases are asymptomatic or pauci-symptomatic and thus unlikely to present for diagnostic testing. Screening tests are more appropriate as a potential tool for reconstructing prevalence or incidence tests, although these are seldom conducted at random. The random and non-random testing samples appear to represent screening and diagnostic testing goals, respectively, and this is part of why they can be expected not to deliver the same prevalence estimates.}

\vspace{5mm}
In the introduction, we now connect random testing to its purpose as a screening tool and non-random testing which generates the case count data as a diagnostic tool.  Given the infrequent nature of random testing and desire for ongoing prevalence estimates, we motivate the second objective of our paper by the question of whether randomized screening tests can be used to address selection bias in the non-random diagnostic testing to estimate prevalence or incidence over time.
\vspace{5mm}

\item {\bf Section 1.1}
\begin{enumerate}
	\item {\it For the first goal of the paper, it might be helpful to link the biases with those outlined in a set of papers from the Harvard Center for Disease Dynamics: \url{https://doi.org/10.1007/s10654-021-00727-7} and DOI:\url{10.1093/aje/kwaa188}}
	\vspace{5mm}

	We now connect with this literature on potential sources of biases in Section 1.1.  We highlight that our work adds a new perspective on this critical issue, from the error decomposition perspective of Meng et al. (2018).
	\vspace{5mm}

	\item {\it For the second goal of the paper, it would be useful to discuss how the proposed method differs from the more traditional epidemiologic method of obtaining a validation sample. As I see it, the chief distinction is that here the “gold standard” is not a different test (although one could argue that the gold standard being used here is a screening test), but rather it is a less biased selection mechanism. Clarifying the links with validation studies could be helpful for epidemiologists \& statisticians alike, as the use of a validation study to correct for selection bias is a fairly novel suggestion.}
	\vspace{5mm}

	We now clarify this point in Section 1.1.  We make it clear that our goal is to use a probability sample to correct for selection bias. We connect this with validation samples to build auxiliary data to address selection bias.

	\vspace{5mm}
	\item {\it Finally, consider linking the arguments you are making with recent work on ‘target validity’ (see: \url{https://doi.org/10.1093/aje/kwy228}). Literature on target validity has largely been focused on specifying the problem of tackling internal and external validity together, but the method you propose here is a concrete attempt to address both types of validity and extends this conversation nicely.}
	\vspace{5mm}

	We now do this at the end of Section 1.1 in order to help statisticians thinking about validity connect our arguments to this literature.
	\vspace{5mm}
\end{enumerate}
\item {\bf Section 2.1: Diagnostic testing}
\begin{enumerate}
	\item {\it Please clarify in the first paragraph the distinction between incubation period (time to symptom onset) and latent period (time to infectiousness). The viral load will be detectable by at least the end of the latent period which for SARS-CoV-2 occurs before the end of the incubation period. However, individuals are unlikely to present for diagnostic testing until after the end of the incubation period.}
	\vspace{5mm}

	We now clarify this distinction between incubation and latent period.
	\vspace{5mm}
	\item {\it Please also clarify the terminology ‘active infection’ which in some instances appears to refer to individuals in the symptomatic period of infection, and in other instances appears to refer to individuals in the infectious period of infection. We know that these are not identical in SARS-CoV-2 infection and that difference is a key reason why screening and diagnostic tests are not easily comparable.}
	\vspace{5mm}

	We now clarify that we define active infection as individuals infected with SARS-CoV-2 who have yet to fall below a threshold for detectable viral load levels by RT-PCR testing.
	% This includes the infectious period as well as a pre-infectious and post-infectious periods of time.


	\vspace{5mm}
	\item {\it Consider adding some additional information on screening tests, particular anything relevant to the random sample conducted in Indiana which is used as the validation set. Were the screening tests conducted using nasopharyngeal swabs (in many places screening tests use nasal swabs instead and often self-collected samples)? Were the molecular tests used to assay the samples the same for screening and diagnostic tests? Differences in sample collection and testing processes could undermine the utility of this random sample as a validation set.}
	\vspace{5mm}

	We clarify that, to the best of our knowledge, both Indiana Department of Health's (IDOH) molecular testing and the random state-wide sample relied primarily on nasopharyngeal swab using the same set of RT-PCR tests. This was primarily due to the random sampling being done by a scientific team working closely with the IDOH.
	\vspace{5mm}
\end{enumerate}
\item {\bf Section 2.2: Publicly available data}
\begin{enumerate}
	\item {\it This section could benefit from a few remarks on why it is important / useful to focus on publicly available testing data. There are certainly distinct statistical issues, if public data are aggregated, but are the selection bias and measurement error issues different between public and government datasets? If so, how do the solutions presented here apply to government datasets? Can similar or the same approaches be used by health departments to assess the pandemic and guide responses?}
	\vspace{5mm}

	We have added Remark 3 in Section 2.2 discussing public versus government data.  Specifically, we clarify that data privacy restrictions enforce publicly released testing and case count data be aggregated to strata-level covariates and have potentially important missing information (i.e., symptom status and indicator of contact with COVID-19 positive individual).
	While these choices impact our data analytic plan, we clarify that the data analytic framework is expressly designed to guide health departments and policy makers to assess the pandemic and guide future responses.

	\vspace{5mm}
	\item {\it A timeline of restrictions and testing eligibility would be a useful addition to Section 2.2.1. Also, what there no time period where testing in Indiana was restricted on the basis of travel history or suspected exposure?}
	\vspace{5mm}

	Section 2.2.1 now has a complete reconstructed timeline of restrictions, which was based on news reports of state-level testing restrictions.
	\vspace{5mm}
\end{enumerate}
\item {\bf Section 2.3: Probabilistic Samples}
\begin{enumerate}
	\item {\it The $\approx 30\%$ response rate in the supposedly random sample requires some remarks. Even if the original list of eligible individuals was random, it seems unlikely that the final sample remains random with such low response. Is it possible to describe any characteristics that may differ between the intended sample and the final sample? Certainly the final random sample differs in many ways from the Census data presented in Table 1. Also, please clarify if the Census data is US total or Indiana specific. }
	\vspace{5mm}

	We now remark on the response rate in Section 5.  Specifically, we cite how Yiannoutsos et al. (2021) handle non-response in prevalence estimation.  We acknowledge the potential for a difference in characteristics between intended and final sample.  We clarify that the Census data is Indiana specific and is presented as a reference to demonstrate the potential difference between intended and final sample.

	\vspace{5mm}
	\item {\it For the Facebook / CMU survey sample, please present the age breakdown of sampled individuals. It is well-known that the age-distribution of Facebook users differs from the general population, so this is unlikely to be a representative sample of the Indiana population. Other demographic information on this sample would be valuable too. (Why is it not included in Table 1?)}
	\vspace{5mm}

	The difference in age breakdown of sampled individuals in the Facebook / CMU survey, now referred to as Delphi's COVID-19 Trends and Impact Survey (CTIS), from the general population is acknowledged in Section 5.  Table 1 now presents estimates of all demographic and covariate information using Delphi CTIS's weighting procedure.  These weights are used in computing probabilities of selection as shown in Equation (4.2) in Section 4.1.1. Estimates for `Fever' and `Household + Case' only use data collected between April 25th and 29th to match the timing of the random survey.  Note that individual-level racial demographics were not provided by CTIS due to data privacy concerns and there was no corresponding question on percent of tests from individuals with a prior positive test.

	\vspace{5mm}
\end{enumerate}
\item {\bf Section 3: Analysis of case count data}
\begin{enumerate}
	\item {\it Please clarify the definition of ‘active infection’ (any infection, vs infectious, vs symptomatic, vs detectable by molecular testing?). This has important implications for the construction of the compartmental model and for the definition of your period prevalence (AIR).}
	\vspace{5mm}

	As stated in answer to Question 3a, we now clarify that we define active infection as individuals infected with SARS-CoV-2 who have yet to fall below a threshold for detectable viral load levels by RT-PCR testing. We discuss how this impacts the compartmental model and calculation of the AIR in answer to Question 9 from Reviewer \#2.

	\vspace{5mm}
	\item {\it Throughout, there is a general issue of too few distinct variable identifiers – for example, using Y to refer to infection status and y to refer to test status will likely lead to confusion.}
	\vspace{5mm}

	We clarify that this notation was chosen to match standard survey statistics notation: $Y_i$ is the true, fixed value for individual $i$ from the population, while $y_i$ is the observed value for individual $i$ from the sample. To help the reader, we now include a notation glossary in Appendix B.
	\vspace{5mm}

	\item {\it The “natural candidate for AIR” presented is the quantity that has more commonly been referred to as the “test positivity rate” (assuming that yi does in fact indicate test status rather than infection status which is obviously unknown). It would be helpful to mention this.}
	\vspace{5mm}

	This is now stated explicitly in Section 3, paragraph 2.
	\vspace{5mm}
\end{enumerate}
\item {\bf Section 3.1: Imperfect testing}
\begin{enumerate}
	\item {\it Here, the value of distinguishing between test status and infection status becomes clearer. By using $P_j$ to mean ‘error status of testing’, the derivations in this section lead to a focus on false positives and false negatives. However, the real quantity of interest both for individuals and for determining prevalence is surely the positive predictive value and the negative predictive value – that is, Pr($Y_j$=1|positive test result) and Pr($Y_j$=0|negative test result). Unlike the FP and FN which (because they are strictly functions of sensitivity and specificity) are only affected by test characteristics, the PPV and NPV are affected also by prevalence. Acknowledging this leads more directly to the benefit of an iterative method for validating the test results (since FP \& FN should not depend on Y or y \& thus not need an iterative method).}
	\vspace{5mm}

	As stated above, the approach benefits from the fact that FP and FN do not depend on $Y$ or $y$ and are only affected by test characteristics. The edited version of Section 3.1 tries to highlight this a bit more clearly.
	% \textcolor{red}{I'm still a bit confused by this point.  Will review further.}
	\vspace{5mm}
	\item {\it Also worth noting that test positivity is not strictly binary (since it’s generally based on an underlying continuous cycle threshold), and so measurement error could potentially lead towards bias away from, or across, the null even in the absence of selection bias.}
	\vspace{5mm}

	Remark 4 is now included to discuss that the binary testing data is generated from an underlying continuous cycle threshold and that the issue of measurement error remains even in the absence of selection bias.
	\vspace{5mm}

	\item {\it The estimator $y = (\bar y_n-FP) /(1-(FP+FN))$ is a standard estimator for correcting for measurement error using FP \& FN estimated from a validation sample, where yn is the mis-measured average \& y is the corrected average. The use of this estimator to iteratively estimate the point prevalence is a good contribution of this paper, but the failure to distinguish between infection \& test status in the variable names somewhat obscures that this is what the author is doing.}
	\vspace{5mm}

	We now more carefully distinguish this and connect with the epidemiologic terminology to help readers from epidemiology.  And yes, this is the standard estimator.  We clarify, as you stated above, that the contribution of this paper was to derive this estimator from a new perspective (i.e., the decomposition perspective which didn't require any Bayesian machinery).
	\vspace{5mm}

	\item {\it Finally, in the section on effective sample size, please clarify why the FP and FN would be dependent on testing capacity. False positive and false negative rates, as well as sensitivity and specificity, are strictly characteristics of a given test – if increased test capacity or burden would alter the sample collection quality then FP \& FN could change. But if increased test capacity simply changes the underlying prevalence in the group tested (because of opening up testing to uninfected individuals), then the FP \& FN should not change, but the PPV and the NPV will change since these are dependent on prevalence.}
	\vspace{5mm}

	We clarify that increasing testing capacity may impact FP and FN rates if altering changes the sample collection quality, e.g., a testing center switches from nasal swab to oropharyngeal swabs or saliva specimen to increase testing capacity will alter measurement error due to differences in the test characteristics.
\end{enumerate}
\item {\bf Section 3.2: Regrettable Rates}
\begin{enumerate}
	\item {\it Here, and throughout the theoretical example, the model is an SIR model. However, we know that SARS-CoV-2 is better represented with an SEIR model, and indeed the applied example later in the manuscript uses an SEIR approach. Since this paper is very particularly addressing an issue with COVID data it would be beneficial to use an SEIR model throughout rather than the simpler SIR model.}
	\vspace{5mm}

	The SIR model in Section 3.2 is replaced by the SEIR model.  We highlight that the SEIR model has been shown to better represent the SARS-CoV-2 dynamics.  Figure 3a presents updated versions of the potential bias in ratio estimators under an SEIR model.   The conclusion remains largely the same. The only difference from the prior manuscript is that the peak in bias prior to the ratio going less than 1 is less pronounced (prior submission with SIR model has an approximate peak of 5 while current manuscript has approximate peak of 2).
	\vspace{5mm}
\end{enumerate}
\item {\bf Section 3.3: Effective Reproduction Number}
\begin{enumerate}
	\item {\it Here again, an SEIR model would be valuable. It is also important to clarify that the effective reproductive number can be defined in a number of subtly different ways, and that these impact both the method of estimation and the interpretation of the effect. See for example Cori et al; Am J Epi, 178(9):1505-12. 2013 and Wallinga \& Tuenis Am J Epi; 160(6):509-16. 2004 for methods that, while similar, estimate subtly different quantities. It would be helpful to clarify exactly what effective reproductive number is being estimated here – from the formulae, I suspect it is the type described by Cori et al rather than Wallinga \& Tuenis.}
	\vspace{5mm}

	We clarify the subtle differences between the definitions of the instantaneous and case reproductive numbers.  We establish that our approach aims to estimate the instantaneous reproduction number, i.e., is closely related to the proposal by Cori et al. (2013)).

	\vspace{5mm}
	\item {\it The Rt estimator will differ if an SEIR model is used, and it is not clear (to me) that the same intuition about the error will hold.}
	\vspace{5mm}

	%https://academic.oup.com/aje/article/189/11/1438/5864525?searchresult=1, plot SIR
	%https://academic.oup.com/aje/article/178/9/1505/89262?searchresult=1, CORI
	%https://www.nature.com/articles/s41598-020-76563-8.pdf?origin=ppub, Althaus approximate

	This estimator is originally motivated by the SIR model and is exact under the poisson likelihood; however, the moment-based estimator has been proposed as a general tool for assessing epidemic trends in real time (Xinhua Yu, Assessing Epidemic Trends in Real Time With a Simple Ratio Plot, Am J of Epid, 189(11), 2020).  The estimator is closely related to the instantaneous reproduction numbers proposed by Cori et. al. (2013), but we now rely on recent work (Heng and Althaus, The approximately universal shapes of epidemic curves in the SEIR model, Sci Reports, 2020) to propose an approximate moment-based estimator in the SEIR setting. Specifically, Heng and Althaus (2020) derive the following approximate formula for the basic reproduction number under SEIR dynamics:
	$$
	R_0 = \frac{1 + \Lambda \left(\frac{1}{\sigma} + \frac{1}{\gamma} \right)}{S_0 + I_0 (1 + \gamma/\sigma)}.
	$$
	where~$\Lambda$ is the epidemic growth rate, and $S_0$/$I_0$ are the initial fractions of the population that are susceptible and infectious respectively.  For $S_0 \approx 1$ and $I_0 \ll 1$, we have the approximate relationship $R_0 \approx 1 + (\frac{1}{\gamma} + \frac{1}{\sigma})\Lambda$.
	, this is an approximate solution which lacks the term $\Lambda^2 \frac{1}{\gamma \sigma}$.  It corresponds to the case of an exponentially distributed generation time with mean duration $1/\gamma + 1/\sigma$, which is the same as the solution for the SIR model assuming an infectious period of $1/\gamma + 1/\sigma$. For realistic growth rates and choices of $\sigma$ and $\gamma$ for the COVID-19 pandemic, this term is negligible and therefore we work with the approximate formula.
	\vspace{5mm}

	\noindent Note that for the SIR model, the basic reproduction number is~$R_0 = 1 + \frac{1}{\gamma} \Lambda$. So estimate of the instantaneous reproduction number is obtained by replacing $\Lambda$ by an estimate of the  instantaneous growth rate $\hat \Lambda_t = \log (K_t / K_{t-1})$.  Then using the approximate relationship between the SEIR model and an SIR model with infectious period $1/\gamma + 1/\sigma$, we arrive at the new moment-based estimator
	$$
    R_t \approx 1 + \left( \frac{1}{\gamma} + \frac{1}{\sigma} \right) \log \left( \frac{K_t}{K_{t-1}} \right).
	$$
	The updated $R_t$ estimator shows that the same intuition about the error holds when we are considering an SEIR rather than an SIR model.  Figure 3b shows the reproduction number estimates and potential biases under SEIR dynamics which shows the same general conclusions hold.  For completeness, we also present the instantaneous reproduction number estimates using the proposal of Cori et. al. (2014) in Section I of the supplementary materials, which shows the same general conclusions.
\end{enumerate}
\item {\bf Section 3.4: Rate comparisons}
\begin{enumerate}
	\item {\it This section would also be more relevant with an SEIR model}
	\vspace{5mm}

	This section has been also updated to consider an SEIR model.  All conclusions remain the same.
	\vspace{5mm}
\end{enumerate}
\item {\bf Section 4.1: Selection propensity estimation}
\begin{enumerate}
	\item {\it The use of weights to correct for sampling bias is an established practice in epidemiology. It would be helpful to discuss how the proposed method differs from survey sampling weights or transportability weights (see Am J Epidemiol 2017 Oct 15;186(8):1010-1014. doi: 10.1093/aje/kwx164.).}
	\vspace{5mm}

	Remark 5 in Section 4.1.1 discusses how the proposal is tied closely to the use of weights to correct for sampling bias.  Specifically, in epidemiology literature, the survey sampling weights and transportability weights require knowledge of the selection mechanism, which is not available in our setting.  Instead, we require auxiliary information in the form of a random sample to estimate the probability of selection. A key difference from prior approaches is the need to estimate time-varying probabilities of selection, e.g., Figure 6 shows how the selection propensity of young individuals changes dramatically over the observation window.
	\vspace{5mm}
	\item {\it This section, and the sections below, also become somewhat confusing to follow because of the multiple different variables which are all denoted $I_j$. Since a chief goal is to describe and correct for errors in the sampling process, it would be much more useful for the reader if there were different notation for each of the different sampling mechanisms being discussed/used (eg, inclusion in the random sample, inclusion in the non-random validation sample, inclusion in the original diagnostic testing data, inclusion in ?? on day x).}
	\vspace{5mm}

	We apologize for the confusion.  We now include distinct notation for each of the sampling mechanisms.  To strike a balance and keep notation at a minimum, we decided to use superscripts $R$ and $NR$ to distinguish between the probability and non-probability samples.  To facilitate comprehension, we have added a notation glossary in Section B of the supplementary materials.
	\vspace{5mm}
	\item {\it In the final formula (eq 4.3), I am not clear on why the weights are applied to both yk and FP (ie why sum(w*(yk-FP)) and not sum(w*yk) – FP)? This is very likely just me not understanding the derivation in Appendix C, but some explanation of the intuition would be helpful.}
	\vspace{5mm}

	Since weights are normalized, these are equivalent, i.e., $\sum w (x_i) FP / \sum w(x_i) = FP$.  We continue to write it this way to keep the formulas compact.
	\vspace{5mm}
	\item {\it The time-varying section seems somewhat unnecessary since it is simply taking the expectation of the data over some time period – the daily data also does this, but where the time-period is 1 day. }
	\vspace{5mm}

	Section 4.1.3 allows researchers to choose to go beyond simpler moving window averages (i.e., daily/weekly averages).  The kernel function allows the user to choose weighted averages that may be more robust than the uniform kernel (which is equivalent to the a moving average over some fixed window of time).  The proposed approach helps smooth the estimated propensity over time; moreover, the flexibility of the approach for government agencies who may observed random samples at different frequencies is of practical importance.
	\vspace{5mm}
\end{enumerate}
\item {\bf Section 4.2: model-based estimation}
\begin{enumerate}
	\item {\it The SEIR model described in the text and in the supplement differ in a way that I find confusing – the code and equations describe a simple SEIR model but the written text describes an SEmInR model with stratum-specific covariates which appear to apply to at least the I compartments and possibly the E compartments. From the code and equations, presumably the assumption is that all strata mix with identical probability and all strata experience the same probabilities of infection, exposure, and transmission. However, these assumptions are (1) unlikely to hold and (2) make it hard to understand why the model includes strata at all. }
	\vspace{5mm}

	We thank the reviewer for catching this issue.  We now clarify that the model proposed does not assume equal mixing.  Due to limited death data, we consider a population-level SEIR model and extract new infection counts at each time~$\{ I^{new}_t \}$. We then generate strata-specific new infection counts at each time~$\{ I^{new}_{t,k} \}$ via a Multinomial distribution given the new infection counts~$\{ I^{new}_t \}$ and parameters~$\{ p_{t,k} \} $ that has a symmetric Dirichlet prior.  These strata-specific new infection counts are then linked to the observed strata-level death counts via the conditional Poisson model using the infection fatality rate.  We clarify this in Section 5.2 and comment on the code to clarify this more carefully in Appendix H.
	\vspace{5mm}
\end{enumerate}
\item {\bf Section 4.3: Doubly Robust Estimation}
\begin{enumerate}
	\item {\it The term $\mu(x_j)$ in the unnumbered equation does not seem to be defined anywhere – this is presumably coming from the SEIR model given that the rest of the equation is essentially the IPW-based estimator. However, it is unclear to me what output from the SEIR this term includes.}
	\vspace{5mm}

	We clarify this notation in Section 4.3.  Specifically, the SEIR model is used to generate the number of new daily infections which are then used to  calculate the number of individuals with an active infection within each strata.
	\vspace{5mm}
\end{enumerate}
\item {\bf Section 5: Applied analysis}
\begin{enumerate}
	\item {\it The applied example relies heavily on the idea that the random sample is a true random sample of the population of Indiana eligible to be infected with COVID, and that the non-random sample and the Facebook / CMU sample similarly provide representative measures of covariate data. This is not well-established by the short data descriptions given and deserves further emphasis.}
	\vspace{5mm}

	We added Remark 6 to emphasize these potential limitations and highlight how the scientific teams that collected these data sources aimed to address these important issues. A key point is that the case study is to demonstrate our framework for answering the \emph{what to do with this knowledge of selection bias and measurement error} point that was raised by reviewers during the first submission round.  In Remark 6, we urge health policy experts and government officials to assess these issues when applying this framework in their own future work.
	\vspace{5mm}
\end{enumerate}
\item {\bf Section 5.1.1. Disease Prevalence}
\begin{enumerate}
	\item {\it This section seems to assume that individuals can only be tested once but that assumption is unlikely to be true. In particular, people with higher exposures will get tested more frequently than people with lower exposures, and similarly people at higher risk of severe outcomes may be tested more frequently than people with lower risk. Calculating the sampling fraction as the (number of tests)/(total pop – previously tested) thus seems likely to provide a biased sampling fraction estimate.}
	\vspace{5mm}

	Section 5.1.1 studies testing between April 25th to 29th 2020 during which testing was limited.  This led us to the assumption that the individuals who already tested were unlikely to be available for testing as it had only been available since March 2020 and quite limited in supply.  We continue to calculate the sampling fraction as number of tests divided by the total population minus prior tests.  To address the reviewer's concern, we note at the end of Section 5.1.1 that using the total population in the denominator yields a very similar sampling fraction and the calculations change by a negligible amount.
	\vspace{5mm}
	\item {\it Also, it would be useful to clarify whether the random sample, which is being used as the gold standard, was corrected for measurement error, as well as how the issue of 30\% response rate was dealt with in obtaining the prevalence estimate. Given these issues, the random sample estimate could easily also be biased.}
	\vspace{5mm}

	See answers to Questions 5a and 14a above.  The prevalence estimate comes from Yiannoutsos et al. (2021) who detail their adjustments for response rates and measurement error in estimation of active prevalence.  We do not attempt to correct it further, and simply note that (unlike in Meng's case where the ``truth'' is observed) this calculation is approximate and meant to help us understand what the relative sampling rates may be in our setting.
	\vspace{5mm}
\end{enumerate}
\item {\bf Section 5.1.2 Time varying estimate}
\begin{enumerate}
	\item {\it 5.1.2. In the second approach, using hospitalization data to attempt to back-predict fever status seems highly likely to be subject to survivor bias as well as biases driven by differential access to care. Some discussion of the issues with this approach is warranted.}
	\vspace{5mm}

	We thank the reviewer for raising this issue.  We have added a sentence in section 5.1.2 that raises this as a limitation.  We restate here that the need for the imputation approaches is due to the lack of publicly available symptom status information.  We have confirmed that this data is collected by the state, but were informed that the data could not be released due to data privacy / confidentiality issues.  Therefore, as a framework for health policy experts and government officials who have access to the complete government datasets (i.e., not just the public data), these steps would not be necessary.
	\vspace{5mm}
	\item {\it Also, focusing on the non-Hispanic, white male groups seems potentially to limit understanding of the scope of error. Although Indiana has a very high percent of white individuals, the individuals who present for testing (as well as those who test positive, are exposed in the workplace and home, and hospitalized or die) are disproportionately non-white. It is important to show the performance of the method in both white and non-white individuals because non-white individuals are the ones who are experiencing higher rates of illness and death and therefore will likely be more negatively impacted by biased estimates of disease prevalence.}
	\vspace{5mm}

	We acknowledge this shortcoming.  We now include versions of Figure 6 for other groups in Section H.2.1 of the supplementary materials with a discussion of the relative propensities.  Specifically, as the reviewer states above, individuals who identify as Hispanic, Male, and select `Some other race' were twice as likely to present for testing as non-Hispanic, white men -- empirically confirming that our data captures these systematic differences.
	\vspace{5mm}
\end{enumerate}
\item {\bf 5.2 SEIR model}
\begin{enumerate}
	\item {\it As stated above, some description of the assumptions related to contact patterns and transmission between and among age strata is needed. In addition, please clarify which compartments of the model the infection-fatality rate applies to (E \& I or I only), as well as how the model handles death from non-COVID causes from these and other model compartments.}
	\vspace{5mm}

	See our answer to Question 12a above.  In Section 5.2, we now clarify how the infection-fatality rate is used in our model. Similar to Johndrow (2021), the model only uses COVID-19 related deaths.  Deaths to non-COVID causes are not directly modeled but would appear as part of the R component in the population-level SEIR model. Moreover, since the active infection rate is defined as all infected individuals who’s viral load has yet to fall below detectable levels by RT-PCR testing, to construct the active infection estimates we exponentially discount the number of newly infected individuals to estimate the number of individuals with an active infection on each day per strata. In our analysis, the exponential discounting is set to 20 days to match with prior evidence that 30\% to 40\% of people will still test positive at three weeks.

	\vspace{5mm}
	\item {\it Also, the discussion of IFR would benefit greatly from being contextualized to the rest of the paper – the goal of the manuscript is specifically to provide a better estimate of who is infected. Such an estimate would therefore be expected to provide a more valid infection fatality rate, and this in turn implies the IFR used here must be biased. This should be discussed.}
	\vspace{5mm}

   	We contextualize the discussion of IFR in Remark 7 in Section 5.2. We acknowledge that these IFR estimates may be biased.  A sensitivity analysis focused on the issue of choosing an appropriate IFR given the potential biases in available IFR estimates is now performed and presented in Appendix I in the supplementary materials.

	\vspace{5mm}
\end{enumerate}
\end{enumerate}
\newpage

{\bf Response to Reviewer \#2}

I am incredibly grateful for this review.  I agree that the paper has evolved beyond the current title.  This was not updated in the prior submission simply to ensure continuity of presentation (i.e., didn't want to change from first submission).  Here we replace with the title ``Addressing selection bias and measurement error in COVID-19 case count data using auxiliary information''.  We next address the reviewers specific comments.

\begin{enumerate}
\item {\it The paper states some facts about the SARS-CoV-2 virus that are outdated, for example “Upon infection with SARS-CoV-2, an incubation period starts and lasts approximately five days [Lauer et al., 2020] before the viral load is high enough to be detected.” (Page 5). Lauer was based on the original
strain, we see different disease dynamics with the delta strain. Perhaps to be more precise, the author could clarify that this is an example of the original SARS-CoV-2 incubation rather than definitative as there will likely be future variants that will change this further. Similarly, on page 13, the serial
interval of 7 is stated as fact – this is again based on the original strain rather than the current disease dynamics. Stating these numbers as an example rather than definitive would be preferable, I believe.}
\vspace{5mm}

Remark 1 has been added at the end of the introduction to contextualize the manuscript.  Specifically we clarify that our manuscript focuses on COVID-19 case count, testing, and death data collected from April through February 2021 and therefore the numbers presented on disease dynamics are based on the original strain.  While using the framework to account for selection bias and measurement error for data arising from the delta and omicron strains would require additional work, the framework would remain an appropriate tool for this task.


\vspace{5mm}
\item {\it The author states: “If someone has an active infection then after the incubation period, a molecular test with perfect sensitivity will yield a positive result for the next few weeks.” (Page 5) The author states “A molecular test refers to diagnostic tests that aim to detect increased viral load” (page 5) - I think a more precise definition is needed, for example aim to detect viral load above a certain threshold.}
\vspace{5mm}

We thank the reviewer for this point. A more precise definition is provided in Section 2.1.  We also point readers to Mina et al. (2020) where the issue of testing thresholds is discussed in detail.

\vspace{5mm}
\item {\it The author states “If someone has an active infection then after the incubation period, a molecular test with perfect sensitivity will yield a positive result for the next few weeks.” (page 5) I’m not sure it is accurate that the patient will have a positive result for the next few weeks, perhaps “If someone has an active infection then after the incubation period, a molecular test with perfect sensitivity will yield a positive result while the patient has a viral load above the threshold of detection”.}
\vspace{5mm}

We thank the reviewer and now match this more careful language to be more exact in our description found in Section 2.
\vspace{5mm}
\item {\it The author lumps RT-PCR and antigen tests in together and discusses that a patients will likely have positive tests for several weeks while not infectious (able to transmit) – for the most part this is the case for RT-PCR tests, however antigen tests give a clearer picture of infectiousness as the viral threshold is higher and often correlates with the likelihood to transmit.}

\vspace{5mm}
We now make this distinction much clearer.  The section focuses primarily on RT-PCR tests given they are the primary molecular test offered by the Indiana's State Department of Health (ISDH) and thus the basis for the case count and testing data in this paper.
\vspace{5mm}

\item {\it I’m not sure this statement “The primary goal of molecular tests is diagnosis of active infections in the population.” (page 5) is entirely true. I definitely think this is a worthwhile goal, but many in the medical community use molecular tests to determine the diagnosis of an individual patient, regardless if the infection is “active” or not (i.e. whether they are currently infectious rather than infected which is what I assume the author means by “active”).}

\vspace{5mm}
We apologize for the confusion.  See our answer to Question 3b from Reviewer \#1 where we clarify the definition of active infection.  An active infection was meant to refer to individuals who were infected with SARS-CoV-2 who have yet to fall below a threshold for detectable viral load levels by RT-PCR testing.
% This includes the infectious period as well as a pre-infectious and post-infectious periods of time.}

\vspace{5mm}
\item {\it On page 13, the author discusses the bias relative to the “peak” – is this the peak in cases? I believe so, but it would be nice to define a bit more clearly.}

\vspace{5mm}
We clarify in Section 3.2 that this is the peak in the fraction of infected individuals in the population which is presented in Figure 4a.
\vspace{5mm}
\item {\it We would expect the IFR to dramatically change after the introductions of vaccines in December 2020, making this method less applicable without introducing a time-varying component.}

\vspace{5mm}
See our answer to Question 17b of Reviewer \#1. We acknowledge that these IFR estimates may be biased and/or time-varying.  Note the study period ends January 2021 so the bias due to vaccines may be limited.  We now include a sensitivity analysis focused on the issue of choosing an appropriate IFR in Appendix I in the supplementary materials.
% \textcolor{red}{To Do!}

\vspace{5mm}
\item {\it I found Figure 3 in the supplement [confusing]. Figure 3A shows the profile for a 35-44 white male with symptoms and COVID-19 contact and Figure 3B shows the profile for a 65-74 white female without symptoms nor COVID-19 contact – I would have expected the covariates between these (age, ethnicity, race, gender) to be the same rather than differ with the only thing differing being symptoms 1/ COVID-19 contact? As a minor note, the y-axis labels do not match for this plot.}

\vspace{5mm}
We have fixed Figure 3b to match on covariates and have matching y-axes.  See our answer to Question 16b from Reviewer \#1 on additional figures we are including in the revision.  The figure now shows that testing propensity is time-varying, increasing from near 0 in the first few months to a peak of 0.01 around December.  This peak is 8 times lower than the peak in December for the same strata given fever and COVID-19 contact.
\vspace{5mm}
\item {\it The model based estimates in Figure 8 seem so different from the other estimates that it is potentially implausible and suggests misspecification of the IFR. Is this potentially due to the age distribution of cases and the IFR collapsing choices? Perhaps as a sensitivity analysis different IFR values should be examined}

\vspace{5mm}
See our answer to Question 17b of Reviewer \#1 and Question 7 above. A sensitivity analysis focused on the issue of choosing an appropriate IFR is now performed in Appendix I in the supplementary materials.  We also note that the active infection estimate being very different for the model-based method in the prior manuscript was a result of how we decided to define the active infection rate using the SEIR model.  With our more precise definition (see answer to Question 5 above), we now have estimates that are more in line with the IPW estimates.  So while there is likely bias, this bias is likely less pronounced than one would have thought using the calculations from our prior submission.
% \textcolor{red}{To Do!}
\vspace{5mm}
\item {\bf Minor comments:}
	\begin{enumerate}
		\item {\it Figure 3 mention “$\beta = 1.2$ (black)” – I’m not sure what “black” refers to?}
		\vspace{5mm}

		We have removed this and simply state the SEIR parameter values ($\beta = 1.2, \gamma = 0.15, \sigma = 0.3$) underlying the Figure.

		\item {\it In the supplement, page 21, the author refers to “Figure 6a and 6a” – I believe they mean “Figure 6a and 6b”. This also claims to show the likelihood for all age ranges - I believe it is only for males.}
		\vspace{5mm}

		We have fixed this reference.

	\end{enumerate}
\end{enumerate}

\end{letter}






\end{document}





