%%%%%%%%%%%%%%%%%%%%%%%%%%%%%%%%%%%%%%%%%
% Plain Cover Letter
% LaTeX Template
%
% This template has been downloaded from:
% http://www.latextemplates.com
%
% Original author:
% Rensselaer Polytechnic Institute (http://www.rpi.edu/dept/arc/training/latex/resumes/)
%
%%%%%%%%%%%%%%%%%%%%%%%%%%%%%%%%%%%%%%%%%

%----------------------------------------------------------------------------------------
%	PACKAGES AND OTHER DOCUMENT CONFIGURATIONS
%----------------------------------------------------------------------------------------

\documentclass[11pt]{letter} % Default font size of the document, change to 10pt to fit more text

\usepackage{newcent} % Default font is the New Century Schoolbook PostScript font
%\usepackage{helvet} % Uncomment this (while commenting the above line) to use the Helvetica font

% Margins
\topmargin=-1in % Moves the top of the document 1 inch above the default
\textheight=8.5in % Total height of the text on the page before text goes on to the next page, this can be increased in a longer letter
\oddsidemargin=-10pt % Position of the left margin, can be negative or positive if you want more or less room
\textwidth=6.5in % Total width of the text, increase this if the left margin was decreased and vice-versa

\let\raggedleft\raggedright % Pushes the date (at the top) to the left, comment this line to have the date on the right
\usepackage{amsthm, amsfonts}
\usepackage{xcolor}
\usepackage{hyperref}


\def\Y{{\bf Y}}

\begin{document}

%----------------------------------------------------------------------------------------
%	ADDRESSEE SECTION
%----------------------------------------------------------------------------------------

\begin{letter}{Professor
	Peter Glynn\\
	Editor, {\em Journal of Applied Probability}}

%----------------------------------------------------------------------------------------
%	YOUR NAME & ADDRESS SECTION
%----------------------------------------------------------------------------------------

%\begin{center}
%\large
%\end{center}
%\vfill

\signature{Walter Dempsey\\
University of Michigan\\
Department of Biostatistics\\
1415 Washington Heights\\
Ann Arbor, MI 48103} % Your name for the signature at the bottom

%----------------------------------------------------------------------------------------
%	LETTER CONTENT SECTION
%----------------------------------------------------------------------------------------

\vspace{5mm}

\newpage

{\bf Response to Referee}

We appreciate the comments from the referee which has helped improve the paper. We have addressed the concerns as best as possible in this revision. We provide a further point-by-point explanation below. Any comments in {\it italics} indicate a statement from the referee report.

\begin{enumerate}
\item {\it I appreciate the addition of the remark that states that the dynamics mentioned in this paper refer to the original strain. There are still lines (for example: “Upon infection with SARS-CoV-2, an incubation period (time to symptom onset) starts and lasts approximately five days”) that are more definitive than I would prefer. Perhaps adding in “Upon infection with the original SARS-CoV-2 variant” would be preferable.}

\vspace{5mm}
We agree with this point and have gone through the manuscript to reduce the chances a reader miscontrues our explanations. We aim for clarity in these edits by avoiding definitive statements while not making statements that are too vauge or ambiguous.
\vspace{5mm}

\item {\it The reliance on the random sample with such a low response rate may be subject to unmeasured confounding - perhaps a sensitivity analysis showing the impact of a potential unmeasured confounder could strengthen this point.}

\vspace{5mm}
We have added a sensitivity analysis in Section K of the supplementary materials. Note the difficulty of such an analysis comes from needing to specify a joint distribution of the measured covariates~$X_t$, the potential unmeasured and \emph{time-varying} confounder~$U_t$, the self-selection indicator~$I_t$ and the outcome~$Y_t$.  Here, we extend recent work in the causal inference literature on sensitivity analysis to the self-selection setting in order to avoid direct specification of these distributions.  Instead, we specify a conditional distribution of the self-selection propensity~$\pi_t(X_t, U_t)$ given the estimated propensity~$\pi_t (X_t)$ and use this as the basis for a sensitivity analysis to derive a simple expression for potential bias.  The bias depends on two sensitivity parameters: (1) $\alpha_t$ which measures the confounder-selection relationship strength, and (2) $\delta_t$ which measures the confounder-outcome relationship strength.  We calibrate the selected sensitivity parameter values by analyzing the observed data as if a particular observed covariate were an unobserved confounder.  A complete set of derivations and associated discussion are presented in Section K of the supplementary materials, which we briefly discuss in Section 5.3 of the main manuscript.
\vspace{5mm}

\item {\it On the plots that show the active infection rate estimates, perhaps including a line that is the case counts divided by the population as a floor would be worthwhile (at least in the supplement) to demonstrate that the model-based estimates are at least above this.}

\vspace{5mm}
We have added the floor of weekly case counts divided by population size to Figure 11 in the supplementary materials.  We interpret ``case counts'' in the referee comment above as ``weekly case counts'' so that the floor calculation is relevant for estimating the active infection rate.  Note all estimates are above the floor with the doubly robust being closest to the floor in September.
\vspace{5mm}

\item {\it I found the grey hard to differentiate for the figures that have more than two categories – perhaps varying the shapes would help if trying to keep these grey-scale?}

\vspace{5mm}
We tested a variety of means to differentiate among categories in figures with more than two categories.  We found varied shapes made the plots much harder to understand in settings with significant overlap.  Therefore, we opted to simply remove the gray background.  The white background helps with differentiation while keeping the figures in grey-scale to match AOAS figure guidelines.
\vspace{5mm}

\end{enumerate}

\end{letter}






\end{document}





